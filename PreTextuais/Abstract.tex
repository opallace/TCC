\noindent
In 1994 Peter Shor presented an algorithm that solves the factorization problem of integers and the discrete logarithm in polynomial time by a quantum computer, such problems are the basis of the security of widely used asymmetric cryptography algorithms, such as RSA and ECDH. Due to quantum computing research advances, a concern has arisen regarding the security of data that depend on cryptographic systems compromised by Shor's algorithm. In 2016, the National Institute of Standards and Technology (NIST) started the Post-Quantum Cryptography Standardization program, whose objective is to select asymmetric cryptography and digital signature algorithms resistant to attacks performed by classical and quantum computers, which can be used in conventional computers. Among the algorithms selected for standardization in 2022 is the Module-Lattice-based Key-Encapsulation Mechanism (ML-KEM) algorithm for asymmetric cryptography that will be addressed in this work. This work aims to serve as a study material on the ML-KEM algorithm and its relation to lattices, for computer science academics.\\

\noindent
\textbf{Keywords:} cryptography, asymmetric cryptography, post-quantum cryptography, lattices.