\noindent
Em 1994, Peter Shor apresentou um algoritmo que resolve o problema da fatoração de números inteiros e do logaritmo discreto em tempo polinomial por um computador quântico, tais problemas são a base da segurança de algoritmos de criptografia assimétrica amplamente utilizados, tais como RSA e ECDH. Com o avanço nas pesquisas na área da computação quântica, iniciou-se uma preocupação com relação à segurança dos dados que dependem de sistemas criptográficos comprometidos pelo algoritmo de Shor. Em 2016, o \textit{National Institute of Standards and Technology} (NIST) iniciou o programa \textit{Post-Quantum Cryptography Standardization}, cujo objetivo é selecionar algoritmos de criptografia assimétricos e de assinatura digital resistentes a ataques realizados por computadores clássicos e quânticos, que possam ser utilizados em computadores convencionais. Dentre os algoritmos selecionados para padronização em 2022, está o algoritmo \ac{ML-KEM} para criptografia assimétrica, que é abordado nesse trabalho. Este trabalho visa servir como um material de estudos para acadêmicos de ciência da computação sobre o algoritmo \ac{ML-KEM} e sua relação com reticulados.\\

\noindent
\textbf{Palavras-chave:} criptografia, criptografia assimétrica, criptografia pós-quântica, reticulados.
