\chapter{Conclusão}
\label{cap:consideracoes_parciais}
    O algoritmo de criptografia \ac{ML-KEM} é relevante para manter a segurança dos dados em um futuro onde a computação quântica terá capacidade computacional para quebrar a segurança dos principais algoritmos de criptografia atuais. Em 2016, foi publicado o algoritmo Kyber no programa \ac{PQC} do \ac{NIST}, com o objetivo de ser o principal algoritmo de criptografia de chave pública, este foi escolhido e padronizado em 2023 e renomeado para \ac{ML-KEM}. O \ac{ML-KEM} possui sua segurança baseada na dificuldade de se resolver o problema \ac{MLWE} \cite{module-lwe}. O fato deste problema ser recente implica na carência de materiais de fácil compreensão sobre o assunto, tendo em sua maioria artigos que exigem um alto nível de conhecimento para se compreender. Esta foi uma das dificuldades na elaboração deste trabalho, além da necessidade de conhecimento em diversas áreas como álgebra abstrata, teoria dos números, espaços métricos, álgebra linear e criptologia. 

    Dentre os objetivos específicos propostos no Capítulo 1, a implementação simplificada do algoritmo \ac{ML-KEM}, desenvolvida neste trabalho, fornece uma base para que o leitor possa compreender o funcionamento deste algoritmo, e também, a partir deste, possa desenvolver por conta um modelo mais próximo da versão oficial que envolvem conceitos mais sofisticados de segurança e otimização.

    Os trabalhos futuros que podem ter como base este documento incluem um aprofundamento nos problemas computacionais baseados em reticulados, além de otimizações utilizadas nas operações do \ac{ML-KEM} como \ac{NTT}. Outro conceito recente de criptografia é a criptografia homomórfica, este possui estudos que utilizam variantes do problema \ac{LWE}.